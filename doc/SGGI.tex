\documentclass{documentation}

\title{SGGI Package}

\author{Peter A. Brooksbank}
\address{Bucknell University}
\email{pbrooksb@bucknell.edu}


\version{0.1}
\date{\today}
\copyrightyear{2018}

%-----------------------------------------------------------------------------

\begin{document}

\frontmatter

\dominitoc
\maketitle
\tableofcontents

\mainmatter

\chapter{Introduction}

This documentation describes {\sc Magma} functions that facilitate computation with {\em string groups generated by involutions} (SGGIs). 
Functions that compute with these structures already exist in distributed versions of {\sc Magma}~\cite{Magma}. The purpose of the 
current package is to supplement---and in some cases improve upon---the existing machinery.

\subsection*{Citing SGGI} 
To cite the SGGI package, please use the following\\
\\
Peter A. Brooksbank, \emph{SGGI}, version 0.1, GitHub, 2020. \url{https://github.com/galois60/SGGI}. \\
\\
For AMSRefs:
\begin{verbatim}
\bib{SGGI}{misc}{
   author={Brooksbank, Peter A.},
   title={SGGI},
   publisher={GitHub},
   year={2020},
   edition={version 0.1},
   note={\texttt{https://github.com/galois60/SGGI}},
}
\end{verbatim}

\section{Overview}

\section{Version}

%%%%%
\chapter{Main chapter 1}


Here is sample documentation to demonstrate the latex functions.


\section{Black-box tensors}
A user can specify a tensor by a black-box function that evaluates the required
multilinear map.

\index{Tensor!black-box}
\begin{intrinsics}
Tensor(S, F) : SeqEnum, UserProgram -> TenSpcElt, List
Tensor(S, F) : List, UserProgram -> TenSpcElt, List
Tensor(S, F, Cat) : SeqEnum, UserProgram, TenCat -> TenSpcElt, List
Tensor(S, F, Cat) : List, UserProgram, TenCat -> TenSpcElt, List
\end{intrinsics}

Returns a tensor $t$ and a list of maps from the given frame into vector spaces of the returned frame.
Note that $t$ is a tensor over vector spaces---essentially forgetting all other structure.
The last entry of \texttt{S} is assumed to be the codomain of the multilinear map. 
The user-defined function $F$ should take as input a tuple of elements of the domain and return an element of the codomain.
If no tensor category is provided, the Albert's homotopism category is used.

\begin{example}[BBTensorsFrame]
We demonstrate the black-box constructions by first constructing the dot product $\cdot : \mathbb{Q}^4\times \mathbb{Q}^4\rightarrowtail \mathbb{Q}$.
The function used to evaluate our black-box tensor, \texttt{Dot}, must take exactly one argument.
The argument will be a \texttt{Tup}, an element of the Cartesian product $U_{v}\times \cdots\times U_1$.
Note that \texttt{x[i]} is the $i$th entry in the tuple and not the $i$th coordinate.
\begin{code}
> Q := Rationals();
> U := VectorSpace(Q, 4);
> V := VectorSpace(Q, 4);
> W := VectorSpace(Q, 1);  // Vector space, not the field Q
> Dot := func< x | x[1]*Matrix(4, 1, Eltseq(x[2])) >;
\end{code}

Now we will construct the tensor from the data above.
The first object returned is the tensor, and the second is a list of maps, mapping the given frame into the vector space frame.
In this example, since the given frame consists of vector spaces, these maps are trivial.
Note that the list of maps are not needed to work with the given tensor, we will demonstrate this later. 
\begin{code}
> Tensor([U, V, W], Dot);
Tensor of valence 3, U2 x U1 >-> U0
U2 : Full Vector space of degree 4 over Rational Field
U1 : Full Vector space of degree 4 over Rational Field
U0 : Full Vector space of degree 1 over Rational Field
[*
    Mapping from: ModTupFld: U to ModTupFld: U given by a rule,
    Mapping from: ModTupFld: U to ModTupFld: U given by a rule,
    Mapping from: ModTupFld: W to ModTupFld: W given by a rule
*]
\end{code}

We will provide a tensor category for the dot product tensor, so that the returned tensor is not in the default homotopism category. 
We will use instead the $\{2,1\}$-adjoint category.
While the returned tensor prints out the same as above, it does indeed live in a universe.
The details of tensor categories are discussed in Chapter~\ref{ch:tensor-categories}.
\begin{code}
> Cat := AdjointCategory(3, 2, 1);
> Cat;
Tensor category of valence 3 (<-,->,==) ({ 1 },{ 2 },{ 0 })
> 
> t := Tensor([U, V, W], Dot, Cat);
> t;
Tensor of valence 3, U2 x U1 >-> U0
U2 : Full Vector space of degree 4 over Rational Field
U1 : Full Vector space of degree 4 over Rational Field
U0 : Full Vector space of degree 1 over Rational Field
> 
> TensorCategory(t);
Tensor category of valence 3 (<-,->,==) ({ 1 },{ 2 },{ 0 })
\end{code}
\end{example}




\chapter{Main chapter 2}


%-----------------------------------------------------------------------------

%\appendix

\backmatter


\begin{bibdiv}
\begin{biblist}

\bib{Magma}{article}{
  author={Bosma, Wieb},
  author={Cannon, John},
  author={Playoust, Catherine},
  title={The Magma algebra system. I. The user language},
  journal={J. Symbolic Comput.},
  issue={24},
  year={1997},
  number={3-4},
  pages={235-265}
}

\bib{BL}{article}{
   author={Brooksbank, Peter A.},
   author={Leemans, Dimitri},
   title={Rank reduction of string C-group representations},
   journal={Proc. Amer. Math. Soc.},
}

\end{biblist}
\end{bibdiv}

\printindex

\end{document}
