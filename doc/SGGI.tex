\documentclass{documentation}

\title{SGGI Package}

\author{Peter A. Brooksbank}
\address{Bucknell University}
\email{pbrooksb@bucknell.edu}


\version{0.1}
\date{\today}
\copyrightyear{2020}

%-----------------------------------------------------------------------------

\begin{document}

\frontmatter

\dominitoc
\maketitle
\tableofcontents

\mainmatter

\chapter{Introduction}

This documentation describes {\sc Magma} functions that facilitate computation with {\em string groups generated by involutions} (SGGIs). 
Functions that compute with these structures already exist in distributed versions of {\sc Magma}~\cite{Magma}. The purpose of the 
current package is to supplement---and in some cases improve upon---the existing machinery.

\subsection*{Citing SGGI} 
To cite the SGGI package, please use the following:\\
\\
Peter A. Brooksbank, \emph{SGGI}, version 0.1, GitHub, 2020. \url{https://github.com/galois60/SGGI}. \\
\\
For AMSRefs:
\begin{verbatim}
\bib{SGGI}{misc}{
   author={Brooksbank, Peter A.},
   title={SGGI},
   publisher={GitHub},
   year={2020},
   edition={version 0.1},
   note={\texttt{https://github.com/galois60/SGGI}},
}
\end{verbatim}

\section{Overview}

\section{Version}

%%%%%%%%%%
%%%%%%%%%%
\chapter{String groups generated by involutions}


This package works with SGGIs as a data type. We begin by describing the essential ingredients.
A {\bf string group generated by involutions}, or {\bf SGGI} consists of: 
  \begin{enumerate}
  \item a group $H$; and 
  \item a list $h_1,\ldots,h_m$ of involutions of $H$.
  \end{enumerate}
The properties that make it an SGGI are first that $H=\langle h_1,\ldots,h_n\rangle$, and 
secondly that the involutions satisfy the {\bf string condition}, namely 
\begin{align}
\label{eq:string}
\forall i,j\in[m] && |i-j|>1~\Longrightarrow
h_ih_j=h_jh_i
\end{align}
The integer $m$ is the {\bf rank} of the SGGI. The package uses a data type {\color{blue}{\tt SGGI}} for these objects. 

%%%%%%%%%%
\section{Constructing SGGIs}
\label{sec:construct}
The package works with the two representations of groups that are currently most relevant to computation with SGGIs, namely 
permutation groups and matrix groups.
\medskip

A user can specify a SGGI in several related ways.

%\index{Tensor!black-box}
\begin{intrinsics}
StringGroupGeneratedByInvolutions(S) : SeqEnum[GrpPermElt] -> SGGI
StringGroupGeneratedByInvolutions(S) : SeqEnum[GrpMatElt] -> SGGI
\end{intrinsics}
Provided $S$ is a list of involutions satisfying~\eqref{eq:string} this returns the SGGI with group generated by $S$.
\begin{intrinsics}
StringGroupGeneratedByInvolutions(G, S) : GrpPerm, SeqEnum[GrpPermElt] -> SGGI
StringGroupGeneratedByInvolutions(G, S) : GrpMat, SeqEnum[GrpMatElt] -> SGGI
\end{intrinsics}
Similar, but checks that the specified $G$ is in fact generated by $S$. One can build the SGGI on the defining generators 
of input group $G$ as follows:
\begin{intrinsics}
StringGroupGeneratedByInvolutions(G) : GrpPerm -> SGGI
StringGroupGeneratedByInvolutions(G) : GrpMat  -> SGGI
\end{intrinsics}
Here are some basic access functions for SGGIs.
\begin{intrinsics}
Group(H) : SGGI -> Grp
\end{intrinsics}
gives the underlying group of $H$, and
\begin{intrinsics}
Generators(H) : SGGI -> SeqEnum
\end{intrinsics}
returns the sequence $h_1,\ldots,h_n$ of involutions generating that group.
\begin{intrinsics}
Rank(H) : SGGI -> RngIntElt
\end{intrinsics}
is the rank of $H$, and
\begin{intrinsics}
SchlafliType(H) : SGGI -> SeqEnum[RngIntElt]
\end{intrinsics}
return the Schlafli symbol of $H$, namely the sequence ${\rm ord}(h_ih_{i+1})$ for $i=1,\ldots,n-1$.
\begin{intrinsics}
Print(H) : SGGI 
\end{intrinsics}
displays information about $H$.

%%%
\subsection{Special constructions}
\label{sec:special}
The symmetric group $S_{n+1}$ with generators $(1\;2),\,(2\;3),\,\ldots,(n\;n+1)$ is the most obvious 
construction of a SGGI of rank $n$. The corresponding polytope is the simplex:
\begin{intrinsics}
Simplex(n) : RngIntElt -> SGGI
\end{intrinsics}

\begin{example}[Simplex]
We build the $5$-simplex as an SGGI and access its basic features.
\begin{code}
> H := Simplex(5);
> Group(H);
Permutation group acting on a set of cardinality 6
   (1, 2)
   (2, 3)
   (3, 4)
   (4, 5)
   (5, 6)
> Rank(H);
5
> SchlafliType(H);
[3, 3, 3, 3]
> Print(H);
SGGI with underlying group type GrpPerm.
\end{code}
\end{example}

SGGIs arise naturally as quotients of Coxeter groups.
\begin{intrinsics}
ModularReflectionGroup(L, p) : SeqEnum[RngIntElt], RngIntElt -> SGGI
\end{intrinsics}
Given a list $L=[\ell_1,\ldots,\ell_{n-1}]$ of natural numbers such that the reflection group $G_0$
with Schlafli symbol $L$ is crystallographic, return the reduction of $G_0$ modulo
the specified prime $p$. (Note, a ``0'' entry in $L$ is understood as ``$\infty$''.)

%%%
\subsection{Duals}
\label{sec:duals}
If $H$ is an SGGI of rank $n$ with distinguished involution sequence $[h_1,\ldots,h_n]$, evidently there is a {\bf dual}
SGGI (on the same group) of rank $n$ with involution sequence $[h_n,\ldots,h_1]$.
\begin{intrinsics}
Dual(H) : SGGI -> SGGI
\end{intrinsics}
One can also form the {\bf Petrie dual} of an SGGI:
\begin{intrinsics}
PetrieDual(H) : SGGI -> SGGI
\end{intrinsics}
returns the SGGI (on the same group) of rank $n$ whose involution sequence has $h_3$ replaced by $h_1h_3$.
The Petrie construction can also be used to reduce the rank of an SGGI:
\begin{intrinsics}
PetrieReduction(H) : SGGI -> SGGI
\end{intrinsics}
returns the SGGI of rank $n-1$ on involution sequence $[h_2,h_1h_3,h_4,\ldots,h_n]$. (Note, this may or may not 
have the same underlying group as $H$.)

\begin{example}[Reflection Group]
We build the SGGI that is the reduction modulo 7 of the crystallographic Coxeter group of type $[3,3,\infty,3]$,
together with various duals and rank-reduced versions.
\begin{code}
> H := ModularReflectionGroup([3,3,0,3], 7);
> SchlafliType(H);
[3, 3, 7, 3]
> Generic (Group(H));
GL(5, GF(7))
> D := Dual(H); SchlafliType(D);
[3, 7, 3, 3]
> PD := PetrieDual(H); SchlafliType(PD);
[3, 4, 14, 3]
> PR := PetrieReduction(H); SchlafliType(PR); 
[4, 14, 3]
> Group(PR) eq Group(H);
true
\end{code}
\end{example}

\section{Isomorphism Testing}
\label{sec:construct}


\chapter{String C-groups}


%-----------------------------------------------------------------------------

%\appendix

\backmatter


\begin{bibdiv}
\begin{biblist}

\bib{Magma}{article}{
  author={Bosma, Wieb},
  author={Cannon, John},
  author={Playoust, Catherine},
  title={The Magma algebra system. I. The user language},
  journal={J. Symbolic Comput.},
  issue={24},
  year={1997},
  number={3-4},
  pages={235-265}
}

\bib{BL}{article}{
   author={Brooksbank, Peter A.},
   author={Leemans, Dimitri},
   title={Rank reduction of string C-group representations},
   journal={Proc. Amer. Math. Soc.},
}

\end{biblist}
\end{bibdiv}

\printindex

\end{document}
